\documentclass[oneside]{book}
\usepackage{geometry}
\geometry{letterpaper}
\usepackage[parfill]{parskip} 
\usepackage{graphicx}
\usepackage{setspace}
\usepackage{booktabs}
\usepackage{fancyhdr}
\pagestyle{fancy}
\fancypagestyle{plain}{}

\usepackage[
	backend=biber,
	style=ieee
]{biblatex}
\addbibresource{math.bib}

\usepackage{makeidx}
\makeindex
\usepackage[acronym]{glossaries}
\makeglossaries

\usepackage[toc,page]{appendix}

\usepackage[
	colorlinks,
	linkcolor=blue,
	citecolor=blue,urlcolor=blue
]{hyperref}

\usepackage{listings}
\lstdefinestyle{mystyle}{
    basicstyle=\ttfamily\footnotesize,
    breakatwhitespace=false,         
    breaklines=true,                 
    captionpos=b,                    
    keepspaces=true,                 
    numbers=left,                    
    numbersep=5pt,                  
    showspaces=false,                
    showstringspaces=false,
    showtabs=false,                  
    tabsize=2,
    frame=single
}
\lstset{style=mystyle}

\usepackage{amsmath, amsthm, amssymb, amsfonts}

\newtheorem{definition}{Definition}

\newcommand{\set}[1]{\{#1\}}

\begin{document}

% set the footer 
% clear the footer
\fancyhf{} 
% remove the horizontal line below the header. Remove the line below if you want header
\renewcommand{\headrulewidth}{0pt}
\fancyfoot[L]{Mathematics\\Frequently Asked Questions}
\fancyfoot[C]{\thepage}
%\fancyfoot[R]{}

%\maketitle
\frontmatter
\begin{titlepage}
	\vspace{1 in}
	\centering
	{\LARGE {Mathematics Frequently Asked Questions}\par}
	\vspace{1 cm}
	Author\par 
	\texttt{mike@conlen.org}\par
	\vspace{1 cm}
	\today
\end{titlepage}

\thispagestyle{empty}
\textbf{Author}: Michael Conlen \\
% Contributors go in alphabetical order by last name

\textbf{Contributors}:  \\
\\

% this alternates pipe and the letter L in lower case. In many fonts these look the same 
\begin{table}[ht]
	\begin{tabular}{l l l}
		\toprule
		Version & Date & Notes \\
		\hline
		&& \\
		\bottomrule
	\end{tabular}
	\caption{Change History}\label{tab:history}
\end{table}

\tableofcontents
%\lstlistoflistings

\mainmatter
\begin{spacing}{1.618}
\chapter{Background}
This document is designed to serve as a repository of frequently asked questions about Mathematics. The word "frequently" is relative. 

\chapter{Numbers}
\section{Why is $0.\bar{9}=1$}
There's a fairly simple common sense approach to this question, which is to consider the fraction $\frac{1}{3}=0.\bar{3}$. We know that $3\cdot \frac{1}{3}=1$ and it's easy to see that $3\cdot 0.\bar{3}=0.\bar{9}$ and so $0.\bar{9}=1$. This isn't a formal proof but it's convincing. 

This highlights a critical detail that's useful here; there is more than one way to represent a single number. Most readers will accept that $\frac{1}{3}=0.\bar{3}$. Let's look at an easier example; $\frac{1}{2}=\frac{2}{4}$. Both fractions represent the same value and thus the same number. For fractions, also known as rational numbers\index{numbers!rational} we have an easy way to determine if two fractions are the same. We say that two fractions, $\frac{a}{b}$ and $\frac{c}{d}$ are equal if $a\cdot d=b\cdot c$. 

But what does it mean for two real numbers\index{numbers!real} to be equal? For example; is $1.\bar{9}$ equal to $2$? For two real numbers we say that they are equal if their difference is equal to $0$. So now when we think about $0.\bar{9}$ and $1$ we consider what their difference is. We could write $1-0.\bar{9}$ and attempt to compute the answer but this is hard since one of these has an infinite number of digits. To reason about this formally we must understand what we mean by a real number and what arithmetic on real numbers is. If you're interested in thinking about what a real number really is see Appendix \ref{appendix:numbers_and_arithmetic}. 

Without being too formal we can think about the expression $1-0.\bar{9}$. Obviously it's less than $0.1$ since $0.\bar{9}+0.1=1.0\bar{9}$ and we can similarly that it's less than $0.01$ and $0.001$. In fact, think about any number, no matter how small, we can convince ourselves that the expression $1-0.\bar{9}$ is less than that value. One property of real numbers is that if a number which is greater than or equal to 0 is less than every positive number then that number must be $0$. Another way to say this is that there are no infinitely small real numbers. There are other number systems which have these numbers such as the surreal numbers\index{numbers!surreal} and hyperreal numbers\index{numbers!hyperreal} but those behave differently than real numbers in lots of important ways. 

\appendix
\chapter{Numbers and Arithmetic}\label{appendix:numbers_and_arithmetic}

What does it mean that $3.1415...$ is a number? How about $\frac{1}{2}$ or $0$? Most of us have a really good intuition of what natural numbers like $0,~1,~2,~\dots$ are and the same for integers and rational numbers. Our intuition can break down when we think about real numbers, especially the irrational numbers like $\pi$ and $\sqrt{2}$. 

If you recall from elementary school you may have learned about whole numbers, integers\index{numbers!integers}, rational numbers\index{numbers!rational}, real numbers\index{numbers!real} and irrational numbers\index{numbers!irrational}. In this chapter we'll explore why your teachers thought they were important to learn and why we treat them differently. If we want to explore how we might compute something like $\pi+e$ or $\pi\cdot e$ where the expression has an infinite number digits we want to be precise in what we mean. We will start with natural numbers, then build up to integers, rational numbers then real numbers. The first three sets of numbers are relatively easy. Constructing the real numbers and defining arithmetic on them is relatively much more complicated. 

\section{Natural Numbers}\index{numbers!natural}
So what is a natural number and why is it important? The second question is quite easy: understanding natural numbers is the easiest path to understanding more complicated objects such as irrational numbers. The first question is more interesting. The first math problem most people think about is $1+1=2$. If we want to build up to the sum or product of irrational numbers like $\pi$ or $\sqrt{2}$ then we better be precise about what we mean when we say $1+1=2$. 

If you're reading this I presume you have some idea of what sets, relations and functions are. I'm going to present things backwards, depending on your intuition about these concepts, as a means to motivate why the foundations of numbers and arithmetic are constructed the way they; then we will formally work our way up. 

The first thing we want to do is have a set of numbers to work with.  Those will be the natural numbers, denoted $\mathbb{N}$. Informally they are the number 0 and each subsequent whole number after it; that is, $\mathbb{N}=\set{0, 1, 2, \dots}$. In elementary school we are given an addition table that tells us the sum of all pairs of numbers from 0 through 9 and told to memorize it; then we are given a procedure we can use to add even larger numbers together. This gives us a way to compute the sum of any two natural numbers but we'd like to do more. We'd like to be able to prove things about addition which are true for any two natural numbers. In particular we'd like to be able to prove the associative and commutative properties of addition and multiplication. To prove that holds for all pairs of numbers we need a useful way to express addition and multiplication for any two numbers. 

So how do we do this? 

\subsection{Relations}

The first thing to figure out is what we mean by $=$. Obviously we know what it means, but we want to be precise about what we know when we say that two things are equal. From there we can contrast it by what we mean by $>$ or $\geq$. For a more detailed treatment see \cite{enderton77}. 

\begin{definition}[Relation]\index{relation}
Given a set $X$, a relation $R$ is a set of ordered pairs of elements of $X$. 
\end{definition}

We typically write something like $xRy$, which looks awful until you remember we are trying to get to $1+1=2$. The $=$ is the relation and $1+1$ and $2$ are the values on either side of the relation. 

The notion of a relation doesn't tell us much; but we can think of the relation as the set of pairs of things that are related to each other. An example of a relation is an equivalence relation which might contain elements that look like $(0, 0), (1, 1), \dots$. 

\begin{definition}[Equivalence Relation]\index{relation!equivalence}
Given a set $X$, a relation $R$ is an equivalence relation if the following are true
\begin{itemize}
	\item For all $x$ in $X$, $xRx$.
	\item If $xRy$ then $yRx$.
	\item If $xRy$ and $yRz$ then $xRz$
\end{itemize}
\end{definition}

Let's write that in something that looks more familiar
\begin{itemize}
	\item For all numbers $x$, $x=x$. 
	\item If $x=y$ then $y=x$.
	\item If $x=y$ and $y=z$ then $x=z$. 
\end{itemize}

These three properties should be familiar to you. 

\subsection{Functions}

Equivalence and order relations aren't the only type of relations. Functions can also be relations, assuming that they map values from one set into the same set. 

First we need to define the domain and range of a relation: 

\begin{definition}[domain]\index{domain}
	The domain of a relation $R$ is the set of values $x$ such that $(x, y)$ is in the relation. 
\end{definition}

\begin{definition}[range]\index{range}
	The range of a relation $R$ is the set of values $y$ such that $(x, y)$ is in the relation. 
\end{definition}


\begin{definition}
	A function is a relation $F$ such that for each $x$ in the domain of $F$ there is one and only one $y$ such that $xFy$. 
\end{definition}

Note: we typically write $F(x)=y$ rather than $xFy$. 

\subsection{Peano Axioms}
So now that we know what we mean when we say $=$ and \emph{function} we can talk about natural numbers and arithmetic. There's a set of axioms for natural numbers attributed to Giuseppe Peano called the Peano axioms or sometimes the Dedekind-Peano axioms in reference to Richard Dedekind's contributions. For a more through treatment of Peano Axioms in the context of set theory see \cite{suppes72}. These axioms define the set of natural numbers as we know them $\set{0, 1, 2, \dots}$. 

\begin{definition}[Peano Axioms]
The following axioms define a function $S$, which together with the equivalence relation $=$ define the natural numbers. 
\begin{enumerate}
	\item There exists a natural number $0$. 
	\item For all natural numbers $n$ the number $S(n)$ is also a natural number. 
	\item For all natural numbers $m$ and $n$, if $S(m)=S(n)$ then $m=n$. 
	\item There is no natural number $n$ such that $S(n)=0$. 
\end{enumerate}
\end{definition}

So what does all that mean? It means there's a set of numbers that has 0 and since it has 0 it has 1 and since it has 1 it has 2 and so on. This gives us the set $\set{0, 1, 2, \dots}$ and we usually denote that set $\mathbb{N}$. The only thing left is to define how addition works. In grade school we usually write out a big addition table and memorize it, but we can give a more formal definition here using the tools we just built. 

\begin{definition}{Addition of Natural Numbers}\index{addition}
	Let $a,~b$ be natural numbers, then we define addition by
	\begin{alignat*}{4}
		a+b=\begin{cases}
			a&b=0 \\
			S(a+c)&b=S(c)
		\end{cases}
	\end{alignat*}
\end{definition}


\end{spacing}
\backmatter

%\addcontentsline{toc}{chapter}{Acronyms}
%\printglossary[type=\acronymtype]
%\clearpage

%\addcontentsline{toc}{chapter}{Glossary}
%\printglossary
%\clearpage

\addcontentsline{toc}{chapter}{Bibliography}
\printbibliography
\clearpage

\printindex


\end{document}  